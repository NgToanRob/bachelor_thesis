\chapter*{Introduction}
\addcontentsline{toc}{chapter}{Введение}
\label{ch:intro}

Computer vision is one of the most rapidly developing fields in modern artificial intelligence, with applications spanning a wide range of tasks, from autonomous driving to medical diagnostics and industrial automation systems. One of the critical and complex challenges researchers face is object tracking in underwater environments. Underwater conditions are characterized by complex optical distortions, low image contrast, uneven lighting, and various noise effects, which significantly complicate the process of video data processing and analysis.

The relevance of this topic is driven by the need to create reliable monitoring systems and manage underwater robotic complexes, as well as to improve methods for marine security and scientific research of ocean resources. The application of modern machine learning and neural network techniques to solve the problem of object tracking in underwater environments allows for enhanced accuracy and robustness of algorithms, which opens up new opportunities for the practical implementation of developed solutions.

The goal of this work is to investigate and develop computer vision algorithms capable of efficiently tracking underwater objects under significant visual distortions. To achieve this goal, several tasks need to be accomplished: conducting an analytical review of existing methods and algorithms used for tracking objects in complex visual conditions; preparing and analyzing the UOT100 dataset, which contains over 100 video sequences under various underwater conditions; evaluating the effectiveness of classical machine learning methods compared to modern neural network approaches, including algorithms based on correlation filters and Siamese networks; and developing software in Python using libraries such as OpenCV, PyTorch, and TensorFlow to implement and compare the selected algorithms.

The structure of the work includes an introduction that justifies the relevance of the topic and sets the research objectives, an overview of existing solutions, a theoretical part dedicated to studying the principles of computer vision algorithms, a description of the experimental setup and analysis conducted, as well as conclusions and recommendations for further development of the topic. The research carried out will not only identify the strengths and weaknesses of existing approaches but also propose directions for their improvement in the context of underwater applications. 

Thus, the work aims to address practical problems related to improving the efficiency of object tracking systems in extreme conditions, which is of great significance for the development of autonomous underwater technologies and the implementation of new solutions in marine safety and monitoring.

\endinput